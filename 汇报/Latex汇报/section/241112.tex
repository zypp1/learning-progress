\newpage
\section{2024年11月12日}
\begin{tcolorbox}[cyan]
    \begin{description}
        \item[\large \textbf{任务}]
        \item[1、] 读文献学习代价函数构建方法
        \item[2、] 解决优化轨迹“打圈”的问题
        \item[3、] 文献进展报告为四部分:(1)论文所解决的问题(2)算法(3)结果(4)论文启发 
    \end{description}
\end{tcolorbox}
\subsection{Minimum Snap Trajectory Generation and Control for Quadrotors(ICRA2011)}
\subsubsection{论文所解决的问题}
论文提出了四旋翼无人机在严格约束的三维空间中的控制器设计和轨迹生成方法。
另外,论文对无人机的微分平坦特性进行了详细推导,并提出飞行走廊约束方法。
\subsubsection{论文算法}
(1)\textbf{无人机微分平坦特性}


无人机动力学模型如下:
\begin{subequations}
    \begin{numcases}{}
      m\ddot{\mathbf{r}}=-mg\mathbf{z}_W+u_1\mathbf{z}_B \label{dynamic1}\\
      \dot{\boldsymbol{\omega}}_{BW}=\mathcal{I}^{-1}\left[-\boldsymbol{\omega}_{BW}\times \mathcal{I}\boldsymbol{\omega}_{BW}+\left[\begin{array}{c}u_2\\u_3\\u_4\end{array}\right]\right]\label{dynamic2}
    \end{numcases}
\end{subequations}


平坦输出:$\sigma(t) = \left[x,y,z,\psi \right]^T $。由式\ref{dynamic1}可知,机体坐标系的$Z_B$轴方向$\mathbf{z_B}$满足下式
\begin{equation}\label{zBandt}
    \mathbf{z_B}=\frac{\mathbf{t}}{\left\lVert\mathbf{t} \right\rVert },\mathbf{t}= \left[\ddot{\sigma}_1,\ddot{\sigma}_2,\ddot{\sigma}_3 +g  \right]^T
\end{equation}
给定$\psi$,可以计算出$\mathbf{x}_C = \left[\cos\sigma_4,\sin\sigma_4,0  \right]^T$,由于采用$Z-X-Y$的旋转顺序,
坐标轴\(\mathbf{x}_C\)和\(\mathbf{x}_B\)、\(\mathbf{z}_B\)在一个平面,$\mathbf{y}_B$与$\mathbf{z}_B$和$\mathbf{x}_C$正交。
因此,有
\begin{equation}\label{flatness1}
    \mathbf{y}_B=\frac{\mathbf{z}_B\times\mathbf{x}_C}{\|\mathbf{z}_B\times\mathbf{x}_C\|},\begin{array}{c}\mathbf{x}_B=\mathbf{y}_B\times\mathbf{z}_B\end{array}
\end{equation}
之后可以唯一确定从机体坐标系到世界坐标系的旋转矩阵\(^WR_B=\left[\mathbf{x}_B\ \ \mathbf{y}_B\ \ \mathbf{z}_B\right]\),
通过旋转矩阵可以确定无人机的滚动角\(\phi \)和俯仰角$\theta$。对式\ref{dynamic1}求导得
\begin{equation}\label{equ-3-14}
    m\dot{\mathbf{a}}=\dot{u}_1\mathbf{z}_B+\boldsymbol{\omega}_{BW}\times u_1\mathbf{z}_B
\end{equation}
沿\(\mathbf{z}_B\)投影可得\(\dot{u}_1=\mathbf{z}_B\cdot m\dot{\mathbf{a}} \),
并定义向量\(\mathbf{h}_\omega=\boldsymbol{\omega}_{BW}\times\mathbf{z}_B=\frac m{u_1}(\dot{\mathbf{a}}-(\mathbf{z}_B\cdot\dot{\mathbf{a}})\mathbf{z}_B)\)
为\(\frac m{u_1}\dot{\mathbf{a}}\)在\(X_B-Y_B\)平面上的投影。因此
\begin{equation}
   p=-\mathbf{h}_\omega\cdot\mathbf{y}_B,q=\mathbf{h}_\omega\cdot\mathbf{x}_B\label{equ-3-15}
\end{equation}
另外,\(r\)是\(\boldsymbol{\omega}_{BW}\)在\(\mathbf{z}_B\)方向的分量,
并将\(\boldsymbol{\omega}_{BW}\)分解为\(\boldsymbol{\omega}_{BC}+\boldsymbol{\omega}_{CW}\)能够得出
\begin{equation}\label{equ-3-16}
    r=(\omega_{BC}+\omega_{CW})\cdot\mathbf{z}_B=\omega_{CW}\cdot\mathbf{z}_B=\dot{\psi}\mathbf{z}_W\cdot\mathbf{z}_B
\end{equation}
同样,对式\ref{dynamic1}求二阶导可以得出,角加速度也可以被平坦输出及其有限阶导数的函数表示。

对于控制输入\(u\),由式\ref{dynamic1}、\ref{zBandt}可知,净推力\(u_1=m\|\mathbf{t}\|\);
另外,由于角速度和角加速度是\(z\)及其导数的函数,因此可以通过式\ref{dynamic2}来计算输入\(u_2\),\(u_3\)和\(u_4\)。
综上,四旋翼无人机系统的12个状态和4个控制输入都可以写成平坦输出\(\sigma\)及其导数的形式。

(2)\textbf{轨迹生成}


轨迹表示为$m$段$n$阶多项式轨迹,如下
\begin{equation}
    \sigma_T(t)=\left\{\begin{array}{cc}\sum_{i=0}^n\sigma_{Ti1}t^i&t_0\leq t<t_1\\\sum_{i=0}^n\sigma_{Ti2}t^i&t_1\leq t<t_2\\\vdots\\\sum_{i=0}^n\sigma_{Tim}t^i&t_{m-1}\leq t\leq t_m\end{array}\right.
\end{equation}


优化问题的构建:
代价函数包含位置$\mathbf{r_T}$的$k_r$阶导的平方和偏航角$\psi$的$k_\psi$阶导的平方,如下
\begin{equation}
    \begin{aligned}\label{costfun}
        &\min&\int_{t_0}^{t_m}\mu_r\left|\left|\frac{d^{k_r}\mathbf{r_T}}{dt^{k_r}}\right|\right|^2+\mu_\psi\frac{d^{k_\psi}\cdot\psi_\mathbf{T}}{dt^{k_\psi}}^2dt&\\
        &\text{s.t.}&\sigma_T(t_i)=\sigma_i,&i=0,...,m\\
        & &\frac{d^px_T}{dt^p}|_{t=t_j}=0\text{ or free},&j=0,m; p=1,...,k_r\\
        & &\frac{d^py_T}{dt^p}|_{t=t_j}=0\text{ or free},&j=0,m; p=1,...,k_r\\
        & &\frac{d^pz_T}{dt^p}|_{t=t_j}=0\text{ or free},&j=0,m; p=1,...,k_r\\
        & &\frac{d^p\psi_T}{dt^p}|_{t=t_j}=0\text{ or free},&j=0,m; p=1,...,k_\psi
    \end{aligned}
\end{equation}
其中,$\mu_r$、$\mu_\psi$的作用是使积分量无量纲。由于$u_2,u_3$由位置的四阶导表示,并且$u_4$由偏航角的二阶导表示,因此取$k_r=4,k_\psi=2$
将优化变量$\sigma_{Tij}=[x_{Tij},y_{Tij},z_{T\boldsymbol{i}j},\psi_{Tij}]^T$
写成$4mn\times 1$的向量$\mathbf{c}$,则问题转换为一个QP问题,如下
\begin{equation}
    \begin{array}{cc}\min&c^THc+f^Tc\\\mathrm{s.t.}&Ac\leq b\end{array}
\end{equation}
(3)\textbf{无量纲优化}


式\ref{costfun}中的变量$x_T,y_T,z_T,\psi_T$是解耦的,因此可以将问题分离为四个优化问题。
考虑单一无量纲变量$\tilde{\omega}(\tau)$的优化问题,如式\ref{costfun1}所示。其中$\tau$代表无量纲时间。
\begin{equation}
    \begin{aligned}\label{costfun1}
    \text{min}&\quad\quad\int_0^1\frac{d^k\tilde{w}(\tau)}{d\tau^k}^2d\tau\\
    \text{s.t.}&\quad\quad\tilde{w}(\tau_i)=\tilde{w}_i,&i=0,...,m\\
    \quad&\frac{d^p\tilde{w}(\tau)}{d\tau^p}|_{\tau=\tau_j}=0\text{ or free},&\tau_j=0,1; p=1,...,k
\end{aligned}
\end{equation}
引入时间维度$t=\alpha \tau$和变量$\omega$,定义为$w(t)=w(\alpha\tau)=\beta_1+\beta_2\tilde{w}(\tau)$。
则问题\ref{costfun1}可变换为如下问题
\begin{equation}
\begin{aligned}\label{costfun2}
    &\text{min}&&\frac{\alpha^{2k-1}}{\beta_{2}}\int_{0}^{\alpha}\frac{d^{k}w(t)}{dt^{k}}dt&&\\
    &\text{s.t.}&&w(t_{i})=\beta_{1}+\beta_{2}\tilde{w}_{i},\quad i=1,...,m\\
    &&&\frac{d^{p}w(t)}{dt^{p}}|_{t=t_{j}}=0\mathrm{~or~free},\quad t_{j}=0,\alpha; p=1,...,k
\end{aligned}
\end{equation}
在问题\ref{costfun2}中边界条件在空间维度平移了$\beta_1$缩放了$\beta_2$,在时间维度缩放了$\alpha$。
因此,如果无量纲优化问题\ref{costfun1}的最优解是$\tilde{\omega}^*$,则问题\ref{costfun2}的最优解为
\begin{equation}
    w^*(t)=\beta_1+\beta_2\tilde{w}^*\left(t/\alpha\right)
\end{equation}
\textbf{整体优化思路}:对于问题\ref{costfun},在解耦后分别与问题\ref{costfun2}对应,得到时间放缩尺度$\alpha$与
和每个变量在空间中的变换$\beta_1,\beta_2$的比例关系。每个变量的时间放缩尺度相同,空间放缩尺度可以不同。


1)时间放缩
\begin{equation}
    \mathbf{r}_T^*(t)=\mathbf{\tilde{r}}_T^*(t/\alpha), \psi_T^*(t)=\tilde{\psi}_T^*(t/\alpha).
\end{equation}


2)空间放缩
\begin{equation}
    x_T^*(t)=x_0+(x_1-x_0)\tilde{x}_T^*(t/t_1),
\end{equation}
其中,$\tilde{x}_T^*(0)=0,\tilde{x}_T^*(1)=1$,且$\tilde{y}_T,\tilde{z}_T$也满足同样的关系。